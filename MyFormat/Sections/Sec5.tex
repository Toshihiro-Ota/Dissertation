\begin{comment}
\documentclass[11pt]{article}  % , titlepage
\usepackage{Common/toshi}
\begin{document}
\end{comment}



%%%%%%%%%%%%%%%%%% section 5 %%%%%%%%%%%%%%%%%%%%%%%%%%%%%%%%%
\section{Conclusion}
\label{sec:discussion}

In this thesis, we have discussed the interrelationships
among supersymmetric gauge theory, string theory, and integrable
lattice model. The answer to the question raised in
the introduction ``why integrable model exists?'' has been addressed to
the extra dimensions behind the system.
The key concept of the correspondence between supersymmetric gauge theory
and integrable lattice model is the underlying two-dimensional TQFT equipped
with line operators.
We also found that branes in string theory and string dualities
provide a natural framework in which such a
correspondence may be realized.


Still, our analysis is far from complete.
One of the most important questions is how to incorporate integrable
\emph{field theory} in our framework.
As we have seen, integrable lattice models % in our framework
are realized by line operators in a two-dimensional TQFT.
A natural direction to proceed is, therefore, to
introduce a higher-dimensional defects in a TQFT.
Fortunately, there is such a candidate \cite{Costello:2019tri}.
We know that Costello's four-dimensional Chern-Simons theory can also
be embedded into string theory setup \cite{Costello:2018txb}.
Thus, along these lines, it should be possible to address the problem
into an essentially same argument as ours.
%\cite{Vicedoetc,Yoshidaetc,Bittlestonetc}


Another possible direction is to find \emph{new} integrable
models through Gauge/YBE correspondence and to consider additional defects on them.
In sections \ref{sec:surface} and \ref{sec:line}, we investigated
the correspondence between supersymmetric gauge theories and integrable
lattice models, and elucidated the counterpart of defect operators in
lattice model side.
They are, however, already known L-operators, transfer matrices, and
quantum algebras in the integrable model literature.
Our construction of the correspondence is actually so powerful that
we can ``define'' new integrable lattice models though the correspondence.
One natural direction is to study Wilson-'t Hooft lines for more general gauge group
other than $\SU$ \cite{Hayashi:2020ofu}.
Another possibility is to derive a new quantum algebra through Gauge/YBE correspondence.
A seminal work to define a new integrable lattice model is given by Yamazaki \cite{Yamazaki:2013nra}, and
it is really integrable \cite{Kels:2015bda,Kels:2017toi,Kels:2017vbc}.
Introducing additional defects in gauge theory side, we may find a
new quantum algebra, such as a generalization of Sklyanin algebra.


Lastly, we wish to extend our discussion to the integrability in
the AdS/CFT correspondence.
Brane tiling techniques, which we have exploited in section \ref{sec:surface},
were originally motivated by the AdS/CFT correspondence.
As is well known, $\CN=4$ super Yang-Mills theory in four dimensions admits
an integrable structure and so does its gravity dual.
We hope our framework may shed light on the aspect of the integrability
in the AdS/CFT correspondence, and hopefully provide a new approach to quantum
gravity.



We hope to explore problems listed above in near future,
but there are many more directions and interesting questions to pursue.
In understanding the fundamental questions in physics, we believe that
it is essential to find more sophisticated viewpoint of the non-perturbative dynamics
of quantum field theories, especially from an integrable structure behind the system.


%\subsection*{Future direction}




%%%%%%%%%%%%%%%%%%%%%%%%%%%%%%%%%%%%%%%%%%%%%%%%%%%%%%%%%



\bibliographystyle{Common/utphys}
%\nocite{*}
\bibliography{Common/Ref}



\end{document}